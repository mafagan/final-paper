%\thispagestyle{empty} %ȡ����ǰҳ��
%\chapter[ABSTRACT��Ӣ��ժҪ��]{Abstract}
\begin{tabular}{l l}
Title��& Research on the Abduction of Knowledge Graph\\
Major��& Software Engineering\\
Name��& \\
Supervisor��& \\
\end{tabular}
\vspace{1cm}
%\chapter*{Abstract}
% \markboth{Ӣ~��~ժ~Ҫ}{Ӣ~��~ժ~Ҫ}
\phantomsection
\addcontentsline{toc}{chapter}{Abstract}

%\vspace*{-1cm

%\centerline{\LARGE \textsf{\textbf{Abstract}}}
\centerline{\xiaoerhao \textbf{Abstract}}
\vspace{0.5cm}

The construction of ontology is a foundation of knowledge base research. Constructing ontology needs to acquire domain-related knowledge and describe the domain-related knowledge in order to give consistent vocabulary terms in the field and the rules to describe the relationship between vocabulary terms. On the basis of effective storage of information, to achieve the terminology and axiom prediction, question and answer inquiries and other reasoning tasks. At present, part of the work has focused on the study of ontology automation and semi-automatic construction. However, the current situation is that the imperfections of the knowledge base are difficult to avoid, and some axioms that conform to the objective facts can not be correctly implied. The main work of this paper is to use the technique of ontology inversion to find out why these observations can not be correctly entailed, and to calculate the semantic explanations of the knowledge base to repair.


Abductive diagnosis is an important reasoning mechanism in ontology reasoning. In the abduction of the ontology, limiting the size of the explanatory set is a key factor that must be considered. In order to balance the contradiction between the expression ability of the explanation set and the solution space, this paper puts forward the concept justification pattern according to the characteristics of the justification. The justification pattern can be obtained from the justification, using the justification pattern we can not only limit the number of explanations, but also provide a semantic basis for the explanation of observation. At the same time, we also suggest that the $\subseteq_\mathsf{ds}$ - $\emph{minimal}$ concept further reduces the number of explanatory sets without guaranteeing that semantics are not missing.


Our goal is to calculate all reasonable explanations, reasonable explanations that need to be met, including acceptable, non-trivial, consistent and is $\subseteq_\mathsf{ds}$ - $\emph{minimal}$. In this paper, we provide a complete algorithm to compute all explanations that satisfy the condition. We also use the generated explanation to achieve the repair of the knowledge base. We have implemented the PBA system based on the algorithm proposed in this paper. In the PBA system, we have tested the ability to solve the knowledge base. The experimental results show that the proposed method is effective and reasonable.


{\bf Key Words:  Ontology; Justification; Abductive Diagnosis}
%\clearpage
