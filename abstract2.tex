%\thispagestyle{empty} %ȡ����ǰҳ��
%\chapter[ABSTRACT��Ӣ��ժҪ��]{Abstract}
\begin{tabular}{l l}
Title��& Research on the Abduction of Knowledge Graph\\
Major��& Software Engineering\\
Name��& \\
Supervisor��& \\
\end{tabular}
\vspace{1cm}
%\chapter*{Abstract}
% \markboth{Ӣ~��~ժ~Ҫ}{Ӣ~��~ժ~Ҫ}
\phantomsection
\addcontentsline{toc}{chapter}{Abstract}

%\vspace*{-1cm

%\centerline{\LARGE \textsf{\textbf{Abstract}}}
\centerline{\xiaoerhao \textbf{Abstract}}
\vspace{0.5cm}

Word embedding is a popular research direction in recent years. The purpose of word embedding is to express the semantic information of entities and relationships in the knowledge base through the dense vector of low latitude. Based on the effective storage of information, it can realize entity prediction, link Prediction and other reasoning tasks. At present, most of the work focuses on finding a more efficient model , so that the trained model can achieve better results under the score of benchmark. However, in reality there are often due to the imperfect data set, resulting in some of the objective facts of the link can not be correctly predicted. The main work of this paper is to use the technique of ontology abduction to find out why these links can not be correctly predicted and to calculate the semantic explanation to repair the data set.

Abductive reasoning is an important reasoning mechanism in describing logic, and its purpose is to find the reasons why observations can not be implied by the ontology and provide logical explanations. In the ontology diagnosis of the ontology, limiting the size of the explanatory set is a key factor that must be considered. In order to balance the contradiction between the expression ability of the interpretation set and the solution space, this paper puts forward the concept jutification pattern according to the characteristics of the justification. Determine the set of templates that can be determined collectively, using the decision set template we limit the number of explanations. At the same time, we also suggest that the $\subseteq_\mathsf{ds}$-$\emph{minimal}$ concept further reduces the number of explanatory sets without guaranteeing that semantics are not missing.

Our goal is to calculate all reasonable explanations, reasonable explanations that need to be met, including acceptable, non-trivial, consistent and is $\subseteq_\mathsf{ds}$-$\emph{minimal}$. In this paper, we provide an algorithm to retieve all the solutions that satisfy the condition. Finally, we use the generated triad to repair the knowledge base. We have implemented the PBA system based on the algorithm proposed in this paper. In the PBA system, we have tested the ability to solve the knowledge base. The experimental results show that the proposed method is effective and reasonable.


{\bf Key Words:  Ontology; Word Embedding; Justification; Abduction Diagnosis}
%\clearpage
